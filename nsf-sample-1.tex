\documentclass[11pt]{article} 
% https://github.com/nsf-open/nsf-proposal-latex-samples/blob/master/nsf-sample-1.tex
% The Proposal and Award Policies and Procedures Guide
% (PAPPG: https://www.nsf.gov/publications/pub_summ.jsp?ods_key=pappg)
% mandates, in Chapter 2, section B.2, that the main text should have a font size no less
% than 11 points for *most* typefaces (including Computer Modern Roman and Times (new) Roman).
% 
% Actually, Helvetica (a.k.a. Arial), Palatino and Courier New can drop
% to 10 point font size, according to the PAPPG, but be aware:
% 10-point fonts (whatever the typeface) will promote reader fatigue.
% Reader fatigue never works to the author's advantage.
%
%
% This sample  assumes the pdflatex chain (not the traditional latex | dvips | ps2pdf chain).
% It should function in the traditional chain if you comment out the hyperref package lines.
% 
%\usepackage{times} % uncomment for Times Roman;
%                   % replace "times" with other font names, as desired. 
\usepackage{bm} % If you want bold math, but without the bloat of AMS packages
\usepackage{color} % doh
\usepackage{hyperref} % LaTeX cross references become hyperlinks in pdf output
\usepackage{graphicx} % For figure insertion
%\usepackage{vmargin} % To set page size
\usepackage[letterpaper,margin=1in]{geometry}
%
% Hyperref stuff: WARNING: don't use \href{link}{clickable text} in the Project Description,
% as Resarch.gov doesn't like it. Likewise, avoid any text containing 
% http:, https:, mailto:, etc.
%
\hypersetup{colorlinks=true,linkcolor=blue,urlcolor=blue}
%

%
% Pseudodot: break up "clickable" strings like "Data.gov" which you can
% enter as Data\pseudodot{}gov in your file.
% Use this trick in case research.gov misreads text such as "Data.gov" as a clickable hyperlink
% (even when it isn't).
%
\newcommand{\pseudodot}{{\lower 2.4pt\hbox{$\cdot$}}}
%\def\pseudodot{{\lower 2.4pt\hbox{$\cdot$}}} %% This is the PlainTeX version.
%
% On with the show...
%
\begin{document}
%
% Remove page numbers---Research.gov will paginate the document instead
%
\pagestyle{empty} 
%
% Line spacing: 
% The Proposal and Award Policies and Procedures Guide
% (PAPPG: https://www.nsf.gov/publications/pub_summ.jsp?ods_key=pappg)
% mandates, in Chapter 2, section B.2, that the main text should have no more than
% six lines to the vertical inch. With 72 points per inch, the minimum line skip 
% would be 12 points.
%
\setlength{\baselineskip}{12.6pt} % in text mode
\setlength{\normalbaselineskip}{12.6pt} % in math mode

%
% Actual text follows
%

\section{Introduction\label{sec:intro}}

Lorem ipsum dolor sit amet, consectetur adipiscing elit. Cras orci lectus, fermentum non mauris a, mattis dignissim risus. Quisque ut tellus felis. Quisque luctus arcu ac sem bibendum dignissim. Curabitur at mauris quis arcu cursus hendrerit. Quisque elementum magna nunc, id sodales dui euismod tincidunt. In non volutpat eros. Nulla nec orci eu erat congue pharetra sit amet fermentum nisi. Fusce condimentum nibh nec libero porttitor, eget luctus magna pulvinar. M{\ae}cenas non ipsum vel ante finibus convallis. Phasellus dictum tincidunt cursus.

Sed maximus risus nec eleifend fringilla. Nullam vit{\ae} lectus lorem. Nulla sed ligula vel nisi vestibulum sagittis. Quisque placerat fermentum commodo. Nullam sagittis mauris in purus pretium facilisis. Curabitur convallis mauris quam, sed finibus ex convallis vit{\ae}. Etiam non egestas dolor. Sed auctor non augue quis blandit. Morbi nec est ligula. Donec venenatis, nulla ac ultricies laoreet, erat risus ullamcorper diam, at vehicula velit orci non sem. 

Mauris volutpat id sapien nec tempor. Phasellus malesuada lacus dolor, at euismod dolor porttitor id. {\ae}nean sit amet tincidunt est, a egestas dolor. Vestibulum varius, lacus ut ultricies laoreet, sapien nulla pharetra mauris, quis egestas sem dui vit{\ae} elit. Pellentesque habitant morbi tristique senectus et netus et malesuada fames ac turpis egestas. Ut est leo, elementum eu turpis sed, ultricies vehicula neque. Morbi volutpat ipsum vestibulum magna scelerisque scelerisque. M{\ae}cenas ligula odio, tempus sed leo non, fermentum vestibulum nisi.%
\footnote{Peter Graves, de son vrai nom Peter Duesler Aurness, est un acteur et
    r\'{e}alisateur am\'{e}ricain, n\'{e} le 18 mars 1926 \`{a} Minneapolis dans le Minnesota (\'{E}tats-Unis), et mort le 14 mars 2010 \`{a} Pacific Palisades, à Los Angeles, en Californie (\'{E}tats-Unis). Il est surtout connu pour son r\^{o}le dans la s\'{e}rie Mission Impossible. }

Sed maximus risus nec eleifend fringilla. Nullam vit{\ae} lectus lorem. Nulla sed ligula vel nisi vestibulum sagittis. Quisque placerat fermentum commodo. Nullam sagittis mauris in purus pretium facilisis. Curabitur convallis mauris quam, sed finibus ex convallis vit{\ae}. Etiam non egestas dolor. Sed auctor non augue quis blandit. Morbi nec est ligula. Donec venenatis, nulla ac ultricies laoreet, erat risus ullamcorper diam, at vehicula velit orci non sem.

\subsection{More detail}

The simplest case of a normal distribution is known as the standard normal distribution or unit normal distribution. This is a special case when $\mu =$  and $\sigma = 1$, and it is described by this probability density function:
$$
    \varphi( x ) = \frac{
     \displaystyle e^{- \frac{ x^2}{2}}
     }{
      \displaystyle \sqrt{2 \pi}
     }\, .
$$
The cumulative distribution function (CDF) of the standard normal distribution, usually denoted with the capital Greek letter $\Phi$, is the integral
$$
 \Phi(x) = \frac{1}{\sqrt{2\pi}} \int_{-\infty}^x \! e^{-t^2\!/2}\, dt\, .
$$

\subsection{La ley del rev\'{o}lver}
 
En 1955 fue elegido, ayudado por su gran porte y por su ideolog\'{\i}a republicana, para interpretar al M\'{a}rshal Matt Dillon en la serie \textit{La ley del rev\'{o}lver} (\textit{Gunsmoke}).
En un primer momento los productores de la cadena televisiva CBS pensaron en contratar a John Wayne, pero la idea finalmente no prosper\'{o}. La serie se emiti\'{o} por primera vez en el a\~{n}o 1955, y acompa\~{n}aban a James Arness los actores Dennis Weaver, como el ayudante Chester Goode; Milburn Stone, como el Doctor; y Amanda Blake, como Kitty. Por la serie pasaron como invitados grandes actores de la talla de John Wayne, presentando el primer cap\'{\i}tulo, Bruce Dern, Bette Davis, Warren Oates, Chuck Connors, Charles Bronson, John Carradine, David Carradine, George Kennedy, William Shatner, Leslie Nielsen, Adam West, y muchos m\'{a}s.
En la serie trabaj\'{o} Burt Reynolds, como Quint Asper, el herrero del pueblo.
La serie se emiti\'{o} entre 1955 y 1975 y en 1987 se film\'{o} la película \textit{Gunsmoke: Return to Dodge}, seguida de \textit{Gunsmoke II: The Last Apache} (1990), \textit{Gunsmoke III: To The Last Man} (1992), \textit{Gunsmoke IV: The Long Ride} y \textit{Gunsmoke V: One Man's Justice}, de 1993, las dos \'{u}ltimas secuelas de la serie. 

%% Figure insertion: commented out for now so as to make the file self-contained.
%%
%\begin{figure}
%\centerline{\includegraphics[width=0.6\hsize]{figure-insert.pdf}}
%\caption{Maritime shipping routes through Iowa.}
%\label{fig-Bessel}
%\end{figure}

\section{Another Section}

As hinted in section~\ref{sec:intro}, the binary entropy function can be bounded in terms of the factorial choose function (e.g., \cite[lemma 2.3.5, p.~33]{gray}). To begin, for integers $k$ and $n$, with $k \le n$, we recall the basic definition
$$
{n \choose k} = \frac{n!}{k! \, (n-k)!}\, ,
$$
and the binomial expansion
$$
(x + y)^M = \sum_{i\,=\,0}^M {M\choose i}\, x^i\, y^{M-i}.
$$
With $\delta$ lying between $0$ and $1$, we may set $x = \delta$ and $y = 1-\delta$
to obtain
$$
1 = \sum_{i\,=\,0}^M {M\choose i} \delta^i\, (1 - \delta)^{M-i}.
$$

We denote the binary entropy function
as
$$
H_2(\delta) = -\delta\log_2\delta - (1 - \delta)\log_2(1 - \delta).
$$
We then have, for $0 < \delta \le \frac{1}{2}$, the inequality
\begin{equation}
 \sum_{i\le \delta M} {M\choose i} \le 2^{M H_2(\delta)}.
 \label{eq:ent-ineq}
\end{equation}


To verify, we note that
$$
2^{-M H_2(\delta)} = \delta^{\delta M} (1 - \delta)^{(1 - \delta)M}.
$$
Now, for $0 < \delta < \frac{1}{2}$, the form
$\delta^k\, (1 - \delta)^{M-k}$ increases as $k$ decreases.
As such, for all $i \le \delta M$, we have
$$
\delta^{\delta M}\,(1 - \delta)^{(1 - \delta)M} \le
 \delta^i\, (1 - \delta)^{M-i} .
$$
This gives the chain of inequalities
\begin{eqnarray*}
 1 &=&
 \sum_{i\,=\,0}^M {M\choose i} \delta^i\, (1 - \delta)^{M-i} 
\\
 &\ge& 
 \sum_{i\,\le\,\delta M} {M\choose i} \delta^i\,(1 - \delta)^{M-i}
\\
 &\ge&
\delta^{\delta M}(1 - \delta)^{(1-\delta)M} \sum_{i\,\le\,\delta M} {M\choose i}
\\
 &=&
 2^{-M H_2(\delta)} \sum_{i\,\le\,\delta M} {M\choose i}\, ,
\end{eqnarray*}
thus confirming the claimed inequality (\ref{eq:ent-ineq}).\hfill$\diamond$



\section{Broader Impacts\label{sec:BI}}

Scientific progress comes in all shapes and sizes. Researchers peer at the microscopic gears of genomes, scan the heavens for clues of our origins. They unearth wind-weathered fossils, labor over complex circuitry, guide students through the maze of learning. Disparate fields, researchers and methods united by one thing: potential. Every NSF grant has the potential to not only advance knowledge, but benefit society---what we call broader impacts. Just like the kaleidoscopic nature of science, broader impacts come in many forms. No matter the method, however, broader impacts ensure all NSF-funded science works to better our world.

As noted in the Proposal and Award Policies and Procedures Guide (section II.C.2.d), ``\textit{Broader impacts may be accomplished through the research itself, through the activities that are directly related to specific research projects, or through activities that are supported by, but are complementary to, the project. NSF values the advancement of scientific knowledge and activities that contribute to the achievement of societally relevant outcomes. Such outcomes include, but are not limited to: full participation of women, persons with disabilities, and underrepresented minorities in science, technology, engineering, and mathematics (STEM); improved STEM education and educator development at any level; increased public scientific literacy and public engagement with science and technology; improved well-being of individuals in society; development of a diverse, globally competitive STEM workforce; increased partnerships between academia, industry, and others; improved national security; increased economic competitiveness of the U.S.; use of science and technology to inform public policy; and enhanced infrastructure for research and education. These examples of societally relevant outcomes should not be considered either comprehensive or prescriptive. Proposers may include appropriate outcomes not covered by these examples.}''

\begin{table}
 \begin{center}
  \begin{tabular}{|l|l|r|}
  \hline
   Apples & per unit & \$1.89 \\
  \hline
   Pears & per pound & \$1.99 \\
  \hline
   Bananas & per bunch & \$1.57 \\
  \hline
   Dunces & per unit & Free \\
   \hline
  \end{tabular}
 \end{center}
 \caption{Prices of common commodities.}
 \label{table-prices}
\end{table}

\section{Intellectual Merit}

The control of light propagation by 3D-microstructured optical fibers and temporal index modulation offers novel opportunities to substantially modify the confinement, guiding, dispersive, and nonlinear properties of spaghetti strands. Spaghetti strands have been the backbone of much scientific and technological advancement in recent decades. Many novel ideas in gastonomics as well as in other fields of science can best be explored in the robust platform of spaghetti fibers due to their controllable transverse confinement and low-loss propagation of light. The proposed novel microstructured spaghetti fibers will expand the horizons for such novel explorations. Multimode pasta provides a rather interesting and complex system where the interactions between the nonlinearity, mode-coupling, and microstructured index profiles are largely unexplored and will be fused with organic Marinara sauce.

The project likewise advances meetings and collaborative science by developing ``In Person'' technology that actually allows people to see and hear each other without using a Zoom connection: it instead leverages Photon Transfer Protocol for video, and Acoustic Molecular Diffusion for audio. With this technology, the video rarely if ever freezes up, and audio dropouts are nearly eliminated, giving an experience almost as authentic as High Definition Virtual Reality. Table~\ref{table-prices} documents the economic impact of this technology.

And avoid links like Data{\pseudodot}gov or en.wikipedia.org in your text, for good measure. (Depending on your pdf viewer, the latter may function as a hyperlink; the former should not.)

\section{Results of Prior NSF Support}

The PI's previous awards have led, directly or indirectly, to extreme high-impact events whose monetary figure is incalculable. These include the Loma Prieta earthquake of October 1989, the Katrina hurricane of August 2005, and the six-hour Facebook outage of October 2021.

\section{Conclusions}

Canada once had a Rhinoceros Party that consistently offered tantalizing election promises that spiced up the discourse. These included: providing higher education by building taller schools; instituting English, French and illiteracy as Canada's three official languages; ending crime by abolishing all laws; and adopting the British system of driving on the left, but phasing it in gradually with only buses driving on the left to begin with. In the 1970s the Rhinos offered a package of corruption and incompetence, and claimed that the then ruling Liberal Party stole their party platform.%
\footnote{economist-dot-com/letters/2021/10/09/letters-to-the-editor}

% The References Cited section would normally begin a new page:
%
%\newpage
%
%
% What follows is what BiBTeX would produce in its .bbl file:
%
\begin{thebibliography}{9}

\bibitem{gray}
R. M. Gray,
\textit{Entropy and Information Theory},
Springer, New York, 1990.

\end{thebibliography}

\end{document}